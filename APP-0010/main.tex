\documentclass{ximera}



\author{Anastasiia Holovchenko \and Anna Davis (project advisor)} \title{Central Limit Theorem} 

\begin{document}

\maketitle

\begin{example}
A potato chips company claims that on average their package contains 200 grams of potato chips. Based on the many years of testing, it is known that the population standard deviation is 10 grams and the amounts are normally distributed. Suppose that the company headquarters suspects that the manufacturer is routinely under-filling the packs of chips to cut the production cost. The company headquarters sends an inspector to the plant to find out what’s going on. Please specify the correct graph for the sampling distribution:

\begin{multipleChoice}  
\choice[correct]{Simple Random Sample}  
\choice{\begin{tikzpicture}
% define normal distribution function 'normaltwo'
\def\normaltwo{\x,{4*1/exp(((\x-3)^2)/2)}}

% input y parameter
\def\y{4.4}

% this line calculates f(y)
%\def\fy{4*1/exp(((\y-3)^2)/2)}

% Shade orange area underneath curve.
%\fill [fill=orange!60] (2.6,0) -- plot[domain=0:4.4] (\normaltwo) -- ({\y},0) -- cycle;

% Draw and label normal distribution function
\draw[color=blue,domain=0:6] plot (\normaltwo) node[right] {};

% Add dashed line dropping down from normal.
%\draw[dashed] ({\y},{\fy}) -- ({\y},0) node[below] {$y$};

% Optional: Add axis labels
\draw (-.2,2.5) node[left] {$f_Y(u)$};
\draw (3,-.5) node[below] {$u$};

% Optional: Add axes
\draw[->] (0,0) -- (6.2,0) node[right] {};
\draw[->] (0,0) -- (0,5) node[above] {};
\end{tikzpicture}}  
\choice{\begin{tikzpicture}
% define normal distribution function 'normaltwo'
\def\normaltwo{\x,{4*1/exp(((\x-3)^2)/2)}}

% input y parameter
\def\y{4.4}

% this line calculates f(y)
%\def\fy{4*1/exp(((\y-3)^2)/2)}

% Shade orange area underneath curve.
%\fill [fill=orange!60] (2.6,0) -- plot[domain=0:4.4] (\normaltwo) -- ({\y},0) -- cycle;

% Draw and label normal distribution function
\draw[color=blue,domain=0:6] plot (\normaltwo) node[right] {};

% Add dashed line dropping down from normal.
%\draw[dashed] ({\y},{\fy}) -- ({\y},0) node[below] {$y$};

% Optional: Add axis labels
\draw (-.2,2.5) node[left] {$f_Y(u)$};
\draw (3,-.5) node[below] {$u$};

% Optional: Add axes
\draw[->] (0,0) -- (6.2,0) node[right] {};
\draw[->] (0,0) -- (0,5) node[above] {};
\end{tikzpicture}}  
\choice{\begin{tikzpicture}
% define normal distribution function 'normaltwo'
\def\normaltwo{\x,{4*1/exp(((\x-3)^2)/2)}}

% input y parameter
\def\y{4.4}

% this line calculates f(y)
%\def\fy{4*1/exp(((\y-3)^2)/2)}

% Shade orange area underneath curve.
%\fill [fill=orange!60] (2.6,0) -- plot[domain=0:4.4] (\normaltwo) -- ({\y},0) -- cycle;

% Draw and label normal distribution function
\draw[color=blue,domain=0:6] plot (\normaltwo) node[right] {};

% Add dashed line dropping down from normal.
%\draw[dashed] ({\y},{\fy}) -- ({\y},0) node[below] {$y$};

% Optional: Add axis labels
\draw (-.2,2.5) node[left] {$f_Y(u)$};
\draw (3,-.5) node[below] {$u$};

% Optional: Add axes
\draw[->] (0,0) -- (6.2,0) node[right] {};
\draw[->] (0,0) -- (0,5) node[above] {};
\end{tikzpicture}}  
\choice{\begin{tikzpicture}
% define normal distribution function 'normaltwo'
\def\normaltwo{\x,{4*1/exp(((\x-3)^2)/2)}}

% input y parameter
\def\y{4.4}

% this line calculates f(y)
%\def\fy{4*1/exp(((\y-3)^2)/2)}

% Shade orange area underneath curve.
%\fill [fill=orange!60] (2.6,0) -- plot[domain=0:4.4] (\normaltwo) -- ({\y},0) -- cycle;

% Draw and label normal distribution function
\draw[color=blue,domain=0:6] plot (\normaltwo) node[right] {};

% Add dashed line dropping down from normal.
%\draw[dashed] ({\y},{\fy}) -- ({\y},0) node[below] {$y$};

% Optional: Add axis labels
\draw (-.2,2.5) node[left] {$f_Y(u)$};
\draw (3,-.5) node[below] {$u$};

% Optional: Add axes
\draw[->] (0,0) -- (6.2,0) node[right] {};
\draw[->] (0,0) -- (0,5) node[above] {};
\end{tikzpicture}} 
\end{multipleChoice}  

\end{example}



\begin{example}
Suppose a company manufactures snack bars. The company claims that the average wight of their granola bar is 50 grams and it's known that the population standard deviation is 3. After a routine check a mechanic said that the weight of granola bars is less than 50 grams. The company decided to test whether its true and took a random sample of 36 granola bars from a conveyor and calculated the average weight. Please identify correct graph for the sampling distribution.
\begin{multipleChoice}  
\choice[correct]{}  
\choice{\begin{tikzpicture}
% define normal distribution function 'normaltwo'
\def\normaltwo{\x,{4*1/exp(((\x-3)^2)/2)}}

% input y parameter
\def\y{4.4}

% this line calculates f(y)
%\def\fy{4*1/exp(((\y-3)^2)/2)}

% Shade orange area underneath curve.
%\fill [fill=orange!60] (2.6,0) -- plot[domain=0:4.4] (\normaltwo) -- ({\y},0) -- cycle;

% Draw and label normal distribution function
\draw[color=blue,domain=0:6] plot (\normaltwo) node[right] {};

% Add dashed line dropping down from normal.
%\draw[dashed] ({\y},{\fy}) -- ({\y},0) node[below] {$y$};

% Optional: Add axis labels
\draw (-.2,2.5) node[left] {$f_Y(u)$};
\draw (3,-.5) node[below] {$u$};

% Optional: Add axes
\draw[->] (0,0) -- (6.2,0) node[right] {};
\draw[->] (0,0) -- (0,5) node[above] {};
\end{tikzpicture}}  
\choice{\begin{tikzpicture}
% define normal distribution function 'normaltwo'
\def\normaltwo{\x,{4*1/exp(((\x-3)^2)/2)}}

% input y parameter
\def\y{4.4}

% this line calculates f(y)
%\def\fy{4*1/exp(((\y-3)^2)/2)}

% Shade orange area underneath curve.
%\fill [fill=orange!60] (2.6,0) -- plot[domain=0:4.4] (\normaltwo) -- ({\y},0) -- cycle;

% Draw and label normal distribution function
\draw[color=blue,domain=0:6] plot (\normaltwo) node[right] {};

% Add dashed line dropping down from normal.
%\draw[dashed] ({\y},{\fy}) -- ({\y},0) node[below] {$y$};

% Optional: Add axis labels
\draw (-.2,2.5) node[left] {$f_Y(u)$};
\draw (3,-.5) node[below] {$u$};

% Optional: Add axes
\draw[->] (0,0) -- (6.2,0) node[right] {};
\draw[->] (0,0) -- (0,5) node[above] {};
\end{tikzpicture}}  
\choice{\begin{tikzpicture}
% define normal distribution function 'normaltwo'
\def\normaltwo{\x,{4*1/exp(((\x-3)^2)/2)}}

% input y parameter
\def\y{4.4}

% this line calculates f(y)
%\def\fy{4*1/exp(((\y-3)^2)/2)}

% Shade orange area underneath curve.
%\fill [fill=orange!60] (2.6,0) -- plot[domain=0:4.4] (\normaltwo) -- ({\y},0) -- cycle;

% Draw and label normal distribution function
\draw[color=blue,domain=0:6] plot (\normaltwo) node[right] {};

% Add dashed line dropping down from normal.
%\draw[dashed] ({\y},{\fy}) -- ({\y},0) node[below] {$y$};

% Optional: Add axis labels
\draw (-.2,2.5) node[left] {$f_Y(u)$};
\draw (3,-.5) node[below] {$u$};

% Optional: Add axes
\draw[->] (0,0) -- (6.2,0) node[right] {};
\draw[->] (0,0) -- (0,5) node[above] {};
\end{tikzpicture}}  
\choice{\begin{tikzpicture}
% define normal distribution function 'normaltwo'
\def\normaltwo{\x,{4*1/exp(((\x-3)^2)/2)}}

% input y parameter
\def\y{4.4}

% this line calculates f(y)
%\def\fy{4*1/exp(((\y-3)^2)/2)}

% Shade orange area underneath curve.
%\fill [fill=orange!60] (2.6,0) -- plot[domain=0:4.4] (\normaltwo) -- ({\y},0) -- cycle;

% Draw and label normal distribution function
\draw[color=blue,domain=0:6] plot (\normaltwo) node[right] {};

% Add dashed line dropping down from normal.
%\draw[dashed] ({\y},{\fy}) -- ({\y},0) node[below] {$y$};

% Optional: Add axis labels
\draw (-.2,2.5) node[left] {$f_Y(u)$};
\draw (3,-.5) node[below] {$u$};

% Optional: Add axes
\draw[->] (0,0) -- (6.2,0) node[right] {};
\draw[->] (0,0) -- (0,5) node[above] {};
\end{tikzpicture}} 
\end{multipleChoice}

\end{example}


\begin{example}
Farmer Joe is looking to sell his farm along with animals. Emily is interested in buying the farm. To make the offer sound more attractive the Farmer Joe states that his chicken on average lay 10 eggs per week, while the national average states that chicken normally lay only 6 eggs per week. Emily wants to test Joe’s claim and decides to make an experiment and see herself how many eggs Joe's chickens will produce in 1 week.
\end{example}


\begin{example}
The average SAT score in the US is . However [FICTIONAL_TOWN_FROM_DR.DAVIS_RADIOSHOW] claims that their teaching techniques are so effective that the average IQ score is 110. Imagine hypothetically OSU developed a study technique that increases the IQ of students by 10 points. Meaning, OSU claims that the average IQ at their school is 110. To test their claim, we have randomly selected 40 people at OSU.. Please specify Null and Alternative Hypothesis.

\end{example}

\begin{example}
\end{example}

\begin{enumerate}
\end{enumerate}

\begin{explanation}
\end{explanation}
  
\begin{exploration}
\end{exploration}

\begin{definition}
\end{definition}








\section*{Works Cited} This application was adapted from Section 1.5 of Keith Nicholson's \href{https://open.umn.edu/opentextbooks/textbooks/linear-algebra-with-applications}{\it Linear Algebra with Applications}. (CC-BY-NC-SA)
 
W. Keith Nicholson, {\it Linear Algebra with Applications}, Lyryx 2018, Open Edition, p. 29

\end{document} 