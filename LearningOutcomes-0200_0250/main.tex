\documentclass{ximera}
 
\input{../preamble.tex}
 
\title{About Section 2}
 
\begin{document}
\begin{abstract}
\end{abstract}
 
\maketitle
 
\section*{Learning Outcomes}
After completing this section, students should be able to do the following.
 
\begin{itemize}
\item Be able to distinguish between natural and assignable causes of variation, and be able to give examples of assignable causes.
\item Be able to discuss characteristics of an in-control and out-of-control process.
    \item Be able to explain the use of X-bar and R control charts.
    \item Be able to find the upper and lower control limits for an X-bar chart.
    \item Be able to describe how the sample size affects control limits for an X-bar chart.
    \item Be able to apply Nelson's rules to X-bar charts to determine when a process is not in statistical control.
    
   \end{itemize}

   \section*{Prerequisites}
   It is assumed that students are familiar with the following concepts.
   \begin{itemize}
   \item Sampling distribution
       \item Central Limit Theorem
        \item The Empirical Rule
   \end{itemize}
 
\end{document}