\documentclass{ximera}
\input{../preamble.tex}

\author{Robert Kenney \and Paul Zachlin \and Nicholas Shay}
\title{X-bar Charts} \license{CC BY-NC-SA 4.0}
\begin{document}

\begin{abstract}
We study basic ideas behind the use of control charts in industry.
\end{abstract}
\maketitle

\begin{onlineOnly}
\section*{X-bar Charts}
\end{onlineOnly}

In the last section we introduced two kinds of control charts, x-bar chart and R-chart.  In this section we will 

\subsection*{When is it time to investigate?}

\begin{problem}\label{prob:controlChartWithSlider}
Suppose that a manufacturing process is known to have a normal distribution with a mean $\mu=3.5$, and standard deviation $\sigma=2$.  A random sample of size $n=4$ is collected every hour to monitor the manufacturing process.  The distribution of sample means (sampling distribution) will be normal and $$\mu_{\bar{x}}=\answer{3.5},\quad \sigma_{\bar{x}}=\answer{1}$$

Determine upper and lower control limits ($\pm 3\sigma$), for the $\bar{X}$-control chart for this manufacturing process.

$$LCL=\answer{0.5},\quad UCL=\answer{6.5}$$

The GeoGebra interactive below shows the means of consecutive samples $1$ through $32$ taken over the course of two days.  Use the scroll bar to navigate all samples and answer the questions below.

\begin{onlineOnly}
\begin{center}
\geogebra{jeu5trsd}{950}{650}
\end{center}
\end{onlineOnly}

What can you say about Samples  1-6:
\begin{multipleChoice}
    \choice{Everything looks good, keep the process going.}
    \choice{A point is above or below one of the control limits.}
    \choice[correct]{Two out of three consecutive points are in Zone 3.}
    \choice{Four out of five consecutive points are in Zone 2.}
    \choice{Eight consecutive points are on the same side of the center line.}
    \choice{Rising or falling pattern is exhibited by seven consecutive points.}
\end{multipleChoice}

What can you say about Samples  7-12:
\begin{multipleChoice}
    \choice{Everything looks good, keep the process going.}
    \choice[correct]{A point is above or below one of the control limits.}
    \choice{Two out of three consecutive points are in Zone 3.}
    \choice{Four out of five consecutive points are in Zone 2.}
    \choice{Eight consecutive points are on the same side of the center line.}
    \choice{Rising or falling pattern is exhibited by seven consecutive points.}
\end{multipleChoice}

What can you say about Samples  13-17:
\begin{multipleChoice}
    \choice{Everything looks good, keep the process going.}
    \choice{A point is above or below one of the control limits.}
    \choice{Two out of three consecutive points are in Zone 3.}
    \choice[correct]{Four out of five consecutive points are in Zone 2.}
    \choice{Eight consecutive points are on the same side of the center line.}
    \choice{Rising or falling pattern is exhibited by seven consecutive points.}
\end{multipleChoice}

What can you say about Samples  18-25:
\begin{multipleChoice}
    \choice{Everything looks good, keep the process going.}
    \choice{A point is above or below one of the control limits.}
    \choice{Two out of three consecutive points are in Zone 3.}
    \choice{Four out of five consecutive points are in Zone 2.}
    \choice[correct]{Eight consecutive points are on the same side of the center line.}
    \choice{Rising or falling pattern is exhibited by seven consecutive points.}
\end{multipleChoice}

What can you say about Samples  26-32:
\begin{multipleChoice}
    \choice{Everything looks good, keep the process going.}
    \choice{A point is above or below one of the control limits.}
    \choice{Two out of three consecutive points are in Zone 3.}
    \choice{Four out of five consecutive points are in Zone 2.}
    \choice{Eight consecutive points are on the same side of the center line.}
    \choice[correct]{Rising or falling pattern is exhibited by seven consecutive points.}
\end{multipleChoice}
\end{problem}

\subsection*{CASE STUDY: Thermoform production}

In theraform production, sheets of plastic are heated and molded into a desired shape.  In the Desmos sheet below, we can see the diameters of 80 different theraform spheres produced in Taiwan.  Notice that the last 20 were too big - something about the production had gone wrong.  Were there warning signs that a problem was coming before it happened?

\desmos{lwyjp4jqzh}{800}{600}

The following YouTube video gives some ideas of how we can use the Desmos sheet.

\youtube{NIg050X6q10}

Also included in the Desmos sheet are $\bar{X}$-control charts for samples of size $n=3$ and $n=5$.  Maybe we can say something about applying Western Electric rules here.


\section*{Practice Problems}

\section*{References}
CASE STUDY on Thermoform Production is modified from:

MIT Open CourseWare \href{https://creativecommons.org/licenses/by-nc-sa/4.0/}{CC-BY-NC-SA}

Control of Manufacturing Processes (SMA 6303)

\href{https://ocw.mit.edu/courses/2-830j-control-of-manufacturing-processes-sma-6303-spring-2008/resources/ps3/}{Assignment 3}, \href{https://ocw.mit.edu/courses/2-830j-control-of-manufacturing-processes-sma-6303-spring-2008/resources/35/}{Part 5}. 

\end{document} 